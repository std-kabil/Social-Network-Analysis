\begin{abstract}
The ``six degrees of separation'' hypothesis suggests that any two people in the world can be connected through a chain of at most six acquaintances. This concept, originating from Stanley Milgram's seminal 1967 experiment, has profound implications for understanding social connectivity, information diffusion, and network resilience. In this project, we empirically investigate this phenomenon using the Pokec social network dataset from the Stanford Network Analysis Project (SNAP), which contains 1,632,803 users and 30,622,564 directed friendship relations from Slovakia's largest online social network.

We focus exclusively on topological analysis, deliberately excluding user profile attributes to maintain computational feasibility on consumer hardware. To study how reciprocity affects network structure, we implement two graph interpretations: (1) an ``all connections'' undirected view treating each directed edge as a bidirectional link, and (2) a ``mutual friendships'' graph retaining only reciprocated edges---analogous to real-world acquaintance chains where both parties acknowledge the relationship.

Our methodology employs a sampling-based pipeline implemented in an interactive Streamlit dashboard. The pipeline extracts the largest connected component (LCC), estimates shortest-path length distributions via Monte Carlo sampling of node pairs, computes network metrics (average degree, density, clustering coefficient), performs Louvain community detection on BFS-sampled subgraphs, and ranks node centrality on manageable samples. We also implement an interactive Network Explorer with community-colored visualization and shortest-path highlighting.

Key findings from the mutual-friendships analysis: the mutual graph contains 8,320,600 reciprocal edges with reciprocity $\rho \approx 0.543$; the LCC encompasses 1,198,274 nodes (73.39\% coverage) and 8,312,834 edges; sampling 10,000 random node pairs yields an average shortest-path length of 5.67 (median 6, maximum observed 11), strongly supporting the small-world hypothesis. Louvain community detection on a 50,000-node BFS sample identifies 37 communities with modularity $Q = 0.7855$, indicating pronounced community structure.

As a machine learning extension, we implement link prediction using graph-based similarity features (Common Neighbors, Jaccard coefficient, Adamic--Adar index, Preferential Attachment) combined with degree-based features, trained via logistic regression. On a balanced dataset of 2,485 positive and 2,485 negative samples from a 1,000-node subgraph, the model achieves ROC-AUC of 0.936 and Average Precision of 0.940, demonstrating that local topological features carry substantial predictive signal.

This work demonstrates that rigorous large-scale network analysis is achievable on resource-constrained systems through careful algorithm selection, strategic sampling, and memory-aware engineering. The complete pipeline, interactive dashboard, and reproducible artifacts are provided for educational and research purposes.
\end{abstract}
