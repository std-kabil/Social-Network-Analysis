\subsection{System implementation and engineering challenges}

This project includes a practical system (Streamlit app) that supports reproducible analysis and interactive exploration. This section describes key engineering challenges and how they were solved.

\subsubsection{Notebooks vs Streamlit}
The initial work started in notebooks, which are good for quick experimentation. However, notebooks can freeze under long-running computations, and it is easy to accidentally create huge intermediate objects. In addition, output artifacts can remain on disk and confuse later results.

Streamlit was introduced to provide a repeatable analysis workflow with interactive parameter control, clear separation between running the analysis and viewing results, and exportable outputs (JSON, CSV, HTML, PNG).

\subsubsection{Memory-driven decisions}
Memory was the main constraint.

\textbf{Mutual edge construction:} A naive mutual-graph approach builds a set of all edges and checks reverse edges. With 30M edges, a full edge-set can freeze a student machine. The implemented solution iterates through directed edges and checks for reverse edges without materializing a giant set.

\textbf{Avoid per-node profile attachment:} Repeated DataFrame lookups for 1.6M nodes are slow and memory-heavy, so the project focuses on topology only.

\textbf{Avoid full-graph heavy algorithms:} Full community detection and betweenness on the full graph are infeasible; sampling and approximations are used.

\subsubsection{UI state and stale outputs}
Streamlit reruns the script frequently. Without session state, it is easy to recompute heavy tasks or display stale results.

The app stores the analysis payload, degree arrays, sampled path lengths, and the visualization subgraph in \texttt{st.session\_state}. The full LCC graph is deleted after extracting needed arrays to reduce memory.

A key issue observed during development was confusing old files in \texttt{outputs/} with the latest run. The Outputs tab was redesigned to separate:
\begin{enumerate}[leftmargin=*]
\item \textbf{Latest run (from session)}: regenerated from session state.
\item \textbf{Files on disk}: shown separately and labeled as potentially legacy.
\end{enumerate}

\subsubsection{Duplicate element IDs}
Streamlit requires unique keys for charts and widgets. The app assigns unique keys (including a per-run \texttt{run\_id}) to prevent \texttt{StreamlitDuplicateElementId} errors and to force regeneration of plots.

\subsubsection{Exporting results}
The app exports JSON/CSV summaries and Plotly HTML plots. It also generates a summary table PNG (when Matplotlib is available). These exports directly support report writing.
